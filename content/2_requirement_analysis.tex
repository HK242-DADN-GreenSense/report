\section{Giới thiệu đề tài}

\subsection{Bối cảnh \& động lực}

Nhà kính được sử dụng để điều chỉnh điều kiện khí hậu cho những cây trồng nhạy cảm hoặc không phải cây bản địa của khí hậu tự nhiên địa phương. Những cây trồng này có thể yêu cầu một môi trường ổn định và kiểm soát chặt chẽ về nhiệt độ, độ ẩm, ánh sáng và các yếu tố khí hậu khác để phát triển tốt. Tuy nhiên, các yếu tố bên ngoài như ánh sáng mặt trời, gió và mưa có ảnh hưởng mạnh mẽ và thay đổi không lường trước được đến nhà kính, khiến việc duy trì môi trường thích hợp luôn là một công việc tốn nhiều công sức và thời gian.

Vì vậy, các nhà kính thương mại hiện đại ngày nay không còn là những cơ sở sản xuất đơn giản mà đã trở thành các cơ sở sản xuất công nghệ cao. Các thiết bị hiện đại như màn che động cơ tự động, hệ thống điều khiển khí hậu nhân tạo, hệ thống chiếu sáng tự động và các cảm biến môi trường đã được tích hợp để giám sát và điều khiển điều kiện bên trong nhà kính một cách chính xác. Những hệ thống này thường được điều khiển bởi máy tính, giúp tự động hóa các quy trình và tối ưu hóa điều kiện sinh trưởng cho cây trồng, từ đó giảm thiểu sự can thiệp thủ công và nâng cao hiệu quả sản xuất.

Bên cạnh đó, các kỹ thuật tiên tiến như cảm biến nhiệt độ, độ ẩm, áp suất và ánh sáng được sử dụng để thu thập dữ liệu liên tục, giúp người dùng có thể phân tích và điều chỉnh điều kiện một cách chủ động. Những dữ liệu này không chỉ phục vụ việc điều khiển trong thời gian thực, mà còn là cơ sở để đánh giá hiệu suất và điều chỉnh chiến lược canh tác trong dài hạn.

\subsection{Mục tiêu của dự án}

Dự án hướng đến việc phát triển một nguyên mẫu hệ thống giám sát và điều khiển nhà kính thông minh, kết hợp các tiến bộ trong công nghệ Internet, IoT (Internet of Things) và kết nối không dây. Mục tiêu chính bao gồm:
\begin{itemize}
    \item Xây dựng một hệ thống có khả năng giám sát các yếu tố môi trường quan trọng bên trong nhà kính như nhiệt độ, độ ẩm và cường độ ánh sáng.
    \item Cho phép người dùng điều khiển từ xa các thiết bị như quạt thông gió, máy bơm nước, đèn chiếu sáng hoặc hệ thống phun sương.
    \item Hiển thị dữ liệu thời gian thực và cung cấp giao diện trực quan để quản lý nhà kính một cách tiện lợi thông qua trình duyệt web hoặc ứng dụng di động.
    \item Góp phần giảm thiểu lao động thủ công, nâng cao hiệu suất sản xuất và tăng độ tin cậy trong quá trình canh tác nông nghiệp công nghệ cao.
\end{itemize}

\subsection{Phạm vi dự án}

Trong phạm vi của dự án này, nhóm sẽ tập trung vào phát triển một nguyên mẫu chức năng cơ bản với các thành phần sau:

\begin{itemize}
    \item Phần cứng: Hệ thống cảm biến để thu thập dữ liệu (nhiệt độ, độ ẩm, ánh sáng), vi điều khiển (như ESP32), các thiết bị điều khiển (quạt, đèn, bơm…).
    \item Phần mềm nhúng: Chương trình chạy trên vi điều khiển để thu thập dữ liệu từ cảm biến và điều khiển các thiết bị đầu ra theo lệnh từ người dùng hoặc theo điều kiện định sẵn.
    \item Hệ thống mạng và lưu trữ: Kết nối không dây qua Wi-Fi để truyền dữ liệu đến máy chủ; lưu trữ dữ liệu cảm biến để phục vụ hiển thị và phân tích.
    \item Giao diện người dùng (Web Dashboard): Hiển thị các thông số môi trường theo thời gian thực, điều khiển từ xa các thiết bị trong nhà kính và cung cấp thông báo trạng thái.
\end{itemize}

Dự án sẽ không đi sâu vào các giải pháp xử lý dữ liệu nâng cao như AI, machine learning hay điều khiển mờ trong giai đoạn đầu, mà chỉ tập trung vào việc xây dựng một hệ thống giám sát và điều khiển thông minh ở mức cơ bản, khả thi và dễ triển khai.
\newpage
%%%%%%%%%%%%%%%%%%%



% \newpage
% %%%%%%%%%%%%%%%%%%%






