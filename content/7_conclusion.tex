\section{Kết luận và hướng phát triển}
\subsection{Kết luận chung}

Dự án phát triển hệ thống giám sát và điều khiển nhà kính thông minh đã hoàn thành các mục tiêu chính đề ra. Hệ thống nguyên mẫu được xây dựng thành công dựa trên sự kết hợp của các công nghệ hiện đại như Internet of Things (IoT), trí tuệ nhân tạo (AI) và giao thức MQTT thông qua nền tảng Adafruit IO. Việc áp dụng các công nghệ này giúp hệ thống có khả năng thu thập dữ liệu thời gian thực và điều khiển các thiết bị trong nhà kính một cách hiệu quả và chính xác.

Giao diện người dùng được thiết kế thân thiện và trực quan, cho phép người dùng dễ dàng giám sát các thông số môi trường như nhiệt độ, độ ẩm, ánh sáng cũng như thực hiện các thao tác điều khiển từ xa thông qua website. Đặc biệt, việc tích hợp tính năng điều khiển bằng ngôn ngữ tự nhiên thông qua AI đã nâng cao trải nghiệm người dùng, giúp thao tác trở nên đơn giản và tiện lợi hơn.

Kiến trúc phần mềm được xây dựng theo mô hình MVC và MCP, đảm bảo tính linh hoạt và mở rộng cho hệ thống trong tương lai. Sự phối hợp nhịp nhàng giữa các thành phần frontend, backend, thiết bị IoT và module điều khiển tự động đã giúp hệ thống vận hành ổn định, giảm thiểu sự can thiệp thủ công và nâng cao hiệu suất quản lý nhà kính.

Kết quả thử nghiệm thực tế cho thấy hệ thống hoạt động ổn định và đáp ứng tốt các yêu cầu kỹ thuật, góp phần giảm tải công việc quản lý thủ công, tối ưu hóa điều kiện khí hậu cho cây trồng, từ đó giúp nâng cao năng suất và chất lượng sản phẩm nông nghiệp trong môi trường nhà kính.

\subsection{Hướng phát triển}

Hiện tại, hệ thống chỉ hỗ trợ hoạt động với một bộ thiết bị duy nhất, sử dụng các feed Adafruit IO đã được cấu hình sẵn. Trong tương lai, dự án sẽ được mở rộng để hỗ trợ đồng thời nhiều bộ thiết bị, giúp tăng khả năng quản lý và giám sát cho các nhà kính có quy mô lớn hoặc nhiều khu vực khác nhau.

Hệ thống cần được bổ sung các tính năng cảnh báo và thông báo tự động đến điện thoại của người dùng, có thể qua tin nhắn SMS hoặc ứng dụng di động. Điều này nhằm kịp thời cảnh báo khi xảy ra sự cố hoặc các điều kiện bất thường trong nhà kính, giúp người quản lý nhanh chóng có biện pháp xử lý phù hợp.

Mở rộng tích hợp thêm các loại cảm biến đa dạng hơn như cảm biến CO2, cảm biến đất, hoặc camera giám sát để cung cấp dữ liệu toàn diện hơn về môi trường nhà kính, từ đó nâng cao hiệu quả điều khiển và chăm sóc cây trồng.

Nâng cao khả năng phân tích dữ liệu và dự báo thông minh dựa trên Machine Learning để đưa ra các khuyến nghị tối ưu hóa điều kiện trồng trọt, giúp tăng năng suất và chất lượng sản phẩm.

Phát triển ứng dụng di động với giao diện thân thiện, cho phép người dùng dễ dàng giám sát và điều khiển nhà kính mọi lúc mọi nơi, đồng thời tích hợp tính năng điều khiển giọng nói dựa trên AI nhằm tăng tính tiện lợi.

Tối ưu hóa giao thức truyền thông và bảo mật hệ thống để đảm bảo dữ liệu được truyền tải an toàn, tránh nguy cơ tấn công hoặc truy cập trái phép.

\newpage
